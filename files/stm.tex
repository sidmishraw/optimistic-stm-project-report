%
% stm.tex
% @author Sidharth Mishra
% @description Generic information about STM
% @copyright 2017 Sidharth Mishra
% @created Thu Dec 07 2017 19:08:49 GMT-0800 (PST)
% @last-modified Thu Dec 07 2017 19:08:49 GMT-0800 (PST)
%

\documentclass[../main]{subfiles}
  
\begin{document}

  \section{Software Transactional Memory}

  \subsection{Definition}

  \par
  Software Transactional Memory (STM) is a new approach that accesses memory objects in a way similar to database transactions \cite{weimerskirch2008software}. STM is a variant of the {\em Transactional Memory} by Herlihy M and Moss JEB and is designed by Nir Shavit and Dan Touitou \cite{weimerskirch2008software}\cite{Shavit1997}. \par

  The STM is a {\em shared object} that behaves like a {\em memory} that supports {\em multiple changes} to its addresses by means of {\em transactions} \cite{Shavit1997}. And, a {\em transaction} is a thread of control that applies a finite sequence of {\em primitive operations} to memory --- primitive operations are {\em read} and {\em write}. Therefore, a transaction is a series of read and write operations that appear as one single atomic operation to other threads/transactions \cite{weimerskirch2008software}. \par

  The operations of the transaction are atomic meaning, either all the actions of a transaction succeed or all of them fail --- there is no partial completion. Even though the transactions of the STM seem similar to the transactions in the database management system (DBMS) context, they are fundamentally different. While database transactions are {\bf ACID} compliant, STM transactions are {\bf ACI*} compliant --- the STM transactions are not {\em durable} because they are dealing with primary memory (RAM) \cite{weimerskirch2008software}. \par

  I use the term ``{\em Optimistic STM}'' to refer to an STM that is {\em wait-free} and {\em non-blocking}.
  An STM implementation is {\em wait-free} if any process that {\em repeatedly} attempts to execute a given transaction terminates successfully after a {\em finite} number of {\em machine steps} \cite{Shavit1997}. And, the STM implementation is {\em non-blocking} if the repeated attempts to execute some transaction by a process implies that some other process with some other transaction will terminate successfully after a finite number of {\em machine steps} in the whole system \cite{Shavit1997}. A process is any thread that invokes the execution of a transaction (or the thread that executes its actions atomically). The {\em main} thread could be envisioned as the most common {\em process}. \par

  \subsection{Design and Working Principle}

  \par
  The design of the STM provided in \cite{Shavit1997} and \cite{weimerskirch2008software} vary but their core structure is similar. Although my STM prototypes draw their core structure from these papers, they deviate quite a bit from the original structure --- more about my designs in the later sections. I took the decision to deviate from the original structure to achieve the APIs I wanted to design in {\em Java} and {\em Go}. Nevertheless, my design and the original design converge because they both follow the same execution workflow. \par

  \subsubsection{Design mentioned in the literature}

  \par
  As per the designs mentioned in \cite{Shavit1997} and \cite{weimerskirch2008software}, the STM has the following data structures:

  \begin{itemize}
    \item {\em Memory} [M], a vector which contains the data stored in the STM,

    \item {\em Ownerships} [O], a vector which determines which {\em transaction} owns the particular memory cell in the {\em Memory} vector.
  \end{itemize}

  \par
  Each process ``i'' keeps a {\em record} pointed to by a {\em Rec\textsubscript{i}} pointer and this record is used to store information about the current transaction it initiated \cite{Shavit1997}\cite{weimerskirch2008software}. The {\em records} have the following fields \cite{Shavit1997}\cite{weimerskirch2008software}: \par

  \begin{itemize}
    \item {\em Size}, is the size of the data set,

    \item {\em Add[]}, a vector that contains the data set addresses in the increasing order,

    \item {\em OldValues[]}, a vector of data set's size whose cells are initialized to {\em Null} at the beginning of every transaction. In case of successful transactions, this vector will contain the former values stored in the involved locations,

    \item {\em Version}, is an integer, initially 0, which determines the instance number of the transaction. This field is incremented every time the process terminates a transaction.
  \end{itemize}

  \subsubsection{My deviations}
  
  \par
  I have introduced the following deviations in my STM prototypes: \par
  \begin{itemize}
    \item I treat each {\em record} as the {\em metadata} of the transaction. So, the {\em Rec\textsubscript{i}} objects become my transactions and their content becomes the transaction's metadata.

    \item The design in \cite{Shavit1997} and \cite{weimerskirch2008software} both used the {\em Add[]} vector to store the addresses of the memory cells the transaction intentds to access --- both for read and write operations. However, I maintain two separate sets named ``readSet'' and ``writeSet'' to store the addresses of the memory cells that the transaction intends to read and write respectively. \par

    The reason for maintaining two separate sets for read and write locations is because of the description of the transaction execution workflow in \cite{weimerskirch2008software}. There is a mention of ``write set'' and ``read set'' but, no supporting definition is provided. Since, there is no deviation between the execution workflows of \cite{Shavit1997} and \cite{weimerskirch2008software}, I went ahead with the two set approach. This seems more natural and declarative than the use of {\em Add[]} vector in \cite{Shavit1997}. Another reason for going with design is that the design in \cite{Shavit1997} and \cite{weimerskirch2008software} are very low level designs --- ones that need to be implemented in C/C++. Because, my choice of languages are relatively higher level than C/C++, my implementation designs have changed accordingly. \par

    \item The {\em OldValues} is a {\em hash-map} instead of a {\em vector} in my prototypes. The mapping is from the memory cell address to the data contained in it.

    \item The {\em Ownerships} is a {\em hash-map} instead of a {\em vector} in my prototypes. The mapping is from the memory cell address to the owner transaction.
  \end{itemize}

  \par
  There are other deviations but, these deviations are influenced by the languages used for implementation. These deviations in the design are discussed in the later sections. \par

\end{document}